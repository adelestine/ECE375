% template created by: Russell Haering. arr. Joseph Crop
\documentclass[12pt,letterpaper]{article}
\usepackage{anysize}
\usepackage{cite}
\usepackage{amsmath,amssymb,amsfonts}
\usepackage{algorithm}
\usepackage[noend]{algpseudocode}
\usepackage{graphicx}
\usepackage{multirow}
\usepackage{listings}
\usepackage{xcolor}


\marginsize{2cm}{2cm}{1cm}{1cm}

\lstset{ framexleftmargin=9mm, frame=shadowbox,tabsize = 4}

\begin{document}

\begin{titlepage}
    \vspace*{4cm}
    \begin{flushright}
    {\huge
        ECE 375 Lab 1\\[1cm]
    }
    {\large
        Introduction to AVR Development Tools
    }
    \end{flushright}
    \begin{flushleft}
    Lab session: 015
    
    Time: 12:00-13:50
    \end{flushleft}
    \begin{flushright}
    Author: Astrid Delestine

    Programming partner: Lucas Plastid 

    \vfill
    \rule{5in}{.5mm}\\
    TA Signature
    \end{flushright}

\end{titlepage}

\section{Introduction}
This is the first Lab in the ECE 375 series and it covers the setup and compilation of an AVR Assembly Program. The student will learn how how to use the sample Basic Bump Bot assembly file and send the binaries to the AVR Microcontroller board. For the second part of the lab the student will be expected to download and compile the included C sample program and from it learn how to configure the I/O ports of the ATmega32U4 Microcontroller. The student will then write their own C program and upload it to the Microcontroller to verify that it runs as expected. The provided programs have been attached in the source code section of this report.

\section{Design}
As for part 1 of this lab assignment, no design needs to be done as the program is supplied. For part 2 of this lab assignment the C program was created to mimic the operaitons of the bump bot assembly file. Firstly the student must understand how the Bump Bot code must operate and they gain this information from the slides provided as they must program the right LED's to illuminate. For our program we decided that we wanted everything to be as readable as possible, thus we created constants for each of the LED directional cues. 
%TODO ADD BLOCK DIAGRAM FOR WHAT IS EXPECTED

\section{Assembly Overview}
As for the Assembly program an overview can be seen below


\subsection{Internal Register Definitions and Constants}
Four different registers have been setup, those being the multipurpose register (mpr), the wait counter register (waitcnt), and two loop counters, for counting the cycles of the delay function. In addition to these, there are several different constants. WTime defines the time  in milliseconds to wait inside the wait loop. The rest of the defined constants are either input bits, engine enable bits, or engine direction bits.

\subsection{Initialization Routine}
The initialization routine sets up several important ports and pointers that allow the rest of the assembly to work. Firstly the stack pointer is initialized at the end of RAM so that when the program pushes and pops items into and out of it, the stack does not interfere with any other data. Port B is then initialized for output, and Port D is initialized for input. The move forward command is also in this phase, to give a default movement type.

\subsection{Main Routine}
The main program constantly checks for if either of the whisker buttons have been hit, by reading the input of the PIND. When one of the whiskers is hit, the correct subroutine is called. As long as no button is hit the bump bot will continue in a straight line.

\subsection{Subroutines}
	\subsubsection{Hit Right}
	The HitRight subroutine describes what happens when the right whisker bit is triggered. The robot will move backwards for a second, then turn left for a second, then it will continue forward. 
	
	\subsubsection{Hit Left}
	The HitLeft subroutine describes what happens when the left whisker bit is triggered. First the bump bot will move backwards for a second, then it will turn right for a second, then it will continue forward. 
	
	\subsubsection{Wait}
	The Wait subroutine controls the wait intervals while the bump bot is preforming an action. Due to each clock cycle taking a measurable amount of time, we can calculate how many times we need to loop for. This function used the olcnt and ilcnt to have two nested loops, running the dec command until they equal zero, thus waiting the requested amount of time. \textbf{The original program was changed by modifying the Wtime constant value by shifting the bit back by 1 space inside of the HitRight subroutine and the HitLeft subroutine. This effectively doubles the wait time. See Lines 167, 201}

\section{C Program Overview}
Each of the methods determined to operate the bump bot can be seen in the code section at the end of this report, their descriptions are here.

\subsection{Definitions and Constants}
Several different constant integer values are prescribed on lines 29 - 33. These constants are the binary values for what the LED's should be when enabled. Several functions are defined here as well, those being, BotActionL() , BotActionR() and goBackwards2Sec(). Each of these are quite self explanatory as to what they do.

\subsection{Main Method}
The main method initializes ports D and B for input and output respectively. Then for port D, due to the fact that it is an input, has its high 4 bits pulled high to enable inputs on those channels. It is importnat to note that all of the inputs are active low, so we must invert them, this is done on line 51. Next the main function enters an infinite while loop, that constantly checks the input of PIND and depending on the inputs, calls the BotActionL() or BotActionR() functions. It ends the while loop by setting the LEDs to forward direction and debouncing the button press by 50ms.

\subsection{Functions}
	\subsubsection{BotActionL()}
	first this function calls the goBackwards2sec() function, then it sets the left motor to forwards and the right motor to backwards, turning the robot right. It then waits 1 second for the action to take place, then returns to the main loop.
	
	\subsubsection{BotActionR()}
	first this function calls the goBackwards2sec() function, then it sets the right motor to forwards and the left motor to backwards, turning the robot left. It then waits 1 second for the action to take place, then returns to the main loop.

	\subsubsection{goBackwards2Sec()}
	This function sets the LED's to the reverse motor direction for two seconds, then returns to the main loop.

\section{Testing}
Testing was only done for the modified bump bot script and for the C program, as the unchanged bump bot script was left alone.
\begin{table}[h]
	\centering
	\begin{tabular}{|l|l|l|ll}
		\cline{1-3}
		Case & Expected & Actual meet expected &  &  \\ \cline{1-3}
	D4 Pressed	&Backward movement$\rightarrow$Turn Left$\rightarrow$Forward&	\checkmark  &  \\ \cline{1-3}
	D5 Pressed	&Backward movement$\rightarrow$Turn Right$\rightarrow$Forward&	\checkmark	&  \\ \cline{1-3}
%		&          &                      &  &  \\ \cline{1-3}
	\end{tabular}
\caption{Assembly Testing Cases}
\end{table}

\begin{table}[h]
	\centering
	\begin{tabular}{|l|l|l|ll}
		\cline{1-3}
		Case & Expected & Actual meet expected &  &  \\ \cline{1-3}
	D4 Pressed	&Backward movement$\rightarrow$Turn Left$\rightarrow$Forward&\checkmark&  &  \\ \cline{1-3}
	D5 Pressed	&Backward movement$\rightarrow$Turn Right$\rightarrow$Forward&\checkmark&  &  \\ \cline{1-3}
%		&          &                      &  &  \\ \cline{1-3}
	\end{tabular}
	\caption{C Testing Cases}
\end{table}

\section{Additional Questions}
\begin{enumerate}
    \item
    Take a look at the code you downloaded for today’s lab. Notice the lines that begin with .def and .equ followed by some type of expression. These are known as pre-compiler directives. Define pre-compiler directive. What is the difference between the .def and .equ directives? (HINT: see Section 5.4 of the AVR Assembler User Guide).

    Pre-compiler directive can be defined as just that, a program or method that is run inside of the compiler, to save certain data values to the memory of the program. these values do not typically change. The .def directive defines a human readable word or reference, that the programmer can use instead of the register directly. This makes the code more human readable. The .equ directive creates a constant variable, that references in this case a number directly. This directive also makes the code more human readable, as one can easily see what number needs to be referenced. The main difference between the two is that .equ defines numbers, while .def defines registers, or places numbers can go.

	\item 
	Read the AVR Instruction Set Manual. Based on this manual, describe the instructions listed below.
	\begin{enumerate}
		\item ADIW
		
		Adds an immiditate value to a word. This is not a text word, rather a binary word. A binary value of 16bits. The value must be from 0 - 63. (pg33 Amtel AVR Instruction Set Manual)
		\item BCLR
		
		Clears a single Flag in the SREG. (pg38 Amtel AVR Instruction Set Manual)
		\item BRCC
		
		Conditional branch if the carry flag is cleared. Tests the carry flag in SREG and if it is zero branches. (pg42 Amtel AVR Instruction Set Manual)
		\item BRGE
		
		Branches by testing the signed flag in SREG, and branches if that flag is cleared. This works with Signed binary numbers. (pg46 Amtel AVR Instruction Set Manual)
		\item COM
		
		Preforms a ones complement operation on the passed register (pg76 Amtel AVR Instruction Set Manual)
		\item EOR
		
		Compares two registers using exclusive or in a bitwise fashion.(pg91 Amtel AVR Instruction Set Manual)
		\item LSL
		
		Preforms a logical shift left on the passed register moving the topmost bit into the carry flag if necessary. (pg120 Amtel AVR Instruction Set Manual)
		\item LSR
		
		Preforms a logical shift right on the passed register and moves the lowest bit into the carry flag if necessary (pg122 Amtel AVR Instruction Set Manual)
		\item NEG
		
		Preforms the two's complement on the passed register, the value \$80 is left unchanged. (pg129 Amtel AVR Instruction Set Manual)
		\item OR
		
		Preforms the logical OR operation between two registers, saves result in the first one. (pg132 Amtel AVR Instruction Set Manual)
		\item ORI
		
		Preforms a logical OR operation between one register and an immediate value. Results in the register (pg133 Amtel AVR Instruction Set Manual)
		\item ROL
		
		Shifts all bits to the left by one place, taking from the carry flag if necessary, and placing the rotated out bit into the carry flag if necessary. (pg143 Amtel AVR Instruction Set Manual)
		\item ROR
		
		Shifts all bits to the right by one place, taking from the carry flag if necessary, then placing the rotated out bit into the carry flag if necessary. (pg145 Amtel AVR Instruction Set Manual)
		\item SBC
		
		Subtracts two registers with the carry flag being subtracted as well if necessary. Places result in first register(pg147 Amtel AVR Instruction Set Manual)
		\item SBIW
		
		Subtracts an immediate value from a word. (pg154 Amtel AVR Instruction Set Manual)
		\item SUB
		
		Subtracts two registers and puts the result in the first register. (pg181 Amtel AVR Instruction Set Manual)
		
	\end{enumerate}
	\item 
	The ATmega32U4 microcontroller has six general-purpose input-output (I/O) ports: Port A through Port F. An I/O port is a collection of pins, and these pins can be individually configured to send (output) or receive (input) a single binary bit. Each port has three I/O registers, which are used to control the behavior of its pins: PORTx, DDRx, and PINx. (The
	“x ” is just a generic notation; for example, Port A’s three I/O registers are PORTA, DDRA, and PINA.)
	\begin{enumerate}
		\item 
		Suppose you want to configure Port B so that all 8 of its pins are configured as outputs. Which I/O register is used to make this configuration, and what 8-bit binary value must be written to configure all 8 pins as outputs?
		
		DDRB would be configured and to enable all 8 pins as outputs it would need to be set to 0b11111111 or \$ff.
		\item 
		Suppose Port D’s pins 4-7 have been configured as inputs. Which I/O register must be used to read the current state of Port D’s pins?
		
		To read from port D's pins, you must use the PINx command, for example PIND
		\item 
		Does the function of a PORTx register differ depending on the setting of its corresponding DDRx register? If so, explain any differences.
		
		Yes it does, due to the fact that if DDRx is set to logical 1, it then uses the PORTx as a voltage Sink or Source. PORTx can only be used. This means that PORTx can only be used as an output port if DDRx is set to a logical 1. For other operations, one would be configuring the pull up resistor.
	\end{enumerate}


    \item
	This lab required you to modify the sample AVR program so the TekBot can reverse for twice as long before turning away and resuming forward motion. Explain how you have done it with reasons.
	
	This has been done by shifting the bit that had been preset as WTime to the left by 1, thus effectively multiplying it by 2. We did this in this way because we wanted to maintain the value of WTime outside of the move backward function. Looking back on the project I believe it would have been easier to just call the Wait method twice.
	
	\item 
	The Part 2 of this lab required you to compile two C programs (one given as a sample, and another that you wrote) into a binary representation that allows them to run directly on your ATmega32U4 board. Explain some of the benefits of writing code in a language like C that can be “cross compiled”. Also, explain some of the drawbacks of writing this way.
	
	Some benefits of having a cross compiled program are that they should be easy to migrate to a different piece of hardware as the code itself should not need to change, only what hardware the code references. Some drawbacks are that it may be difficult to setup for the first time, or it may run slower than a program written specifically for a chip. C however is known for being almost as good as writing directly to hardware, like we are doing with Assembly 
	
	\item 
	The C program you wrote does basically the same thing as the sample AVR	program you looked at in Part 1. What is the size (in bytes) of your Part 1 \& Part 2 output .hex files? Explain why there is a size difference between these two files, even though they both perform the same BumpBot behavior?
	
	The Assembly hex file came out to 485 bytes while the C file came out to 1020 bytes. This is over double the size. The main reason for the discrepancy in size can be attributed to possible bloat or automatic functions built into the C program to make it run smoothly. C works for you in this way, however the Assembly file is smaller due to the fact that we have defined exactly what needs to happen at every step thus making it a smaller file and possibly more precise.
	

\end{enumerate}

\section{Difficulties}
This Lab was quite trivial, as such the only difficulties that we encountered were agreeing on how we would preform the requested task with modifying the original bump bot script. After this programming the bump bot in C was quite simple. As such this lab did not have many difficulties.

\section{Conclusion}
This lab reinforced the ideas of how assembly code is assembled, and how we can make this code run on our boards. Additionally it allowed the students to gain a better understanding of C programming when it relates to bare metal, and programming directly to a chip. 

\section{Source Code}%
\lstinputlisting
[
caption=Assembely Bump Bot Script,
language={[x86masm]Assembler},
numbers =left,
rulesepcolor=\color{blue}
]{../Lab1/Lab1/main.asm}
\lstinputlisting[
caption=C Bump Bot Script,
language=C,
numbers =left,
rulesepcolor=\color{magenta}
]{../Lab1C/Lab1C/main.c}



\end{document}
