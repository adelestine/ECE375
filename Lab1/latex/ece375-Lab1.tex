% template created by: Russell Haering. arr. Joseph Crop
\documentclass[12pt,letterpaper]{article}
\usepackage{anysize}
\usepackage{cite}
\usepackage{amsmath,amssymb,amsfonts}
\usepackage{algorithm}
\usepackage[noend]{algpseudocode}
\usepackage{graphicx}
\usepackage{multirow}
\usepackage{listings}
\usepackage{xcolor}


\marginsize{2cm}{2cm}{1cm}{1cm}

\lstset{ framexleftmargin=9mm, frame=shadowbox,tabsize = 4}

\begin{document}

\begin{titlepage}
    \vspace*{4cm}
    \begin{flushright}
    {\huge
        ECE 375 Lab 1\\[1cm]
    }
    {\large
        Introduction to AVR Development Tools
    }
    \end{flushright}
    \begin{flushleft}
    Lab session: 015
    
    Time: 12:00-13:50
    \end{flushleft}
    \begin{flushright}
    Author: Astrid Delestine

    Programming partner: Lucas Plastid 

    \vfill
    \rule{5in}{.5mm}\\
    TA Signature
    \end{flushright}

\end{titlepage}

\section{Introduction}
This is the first Lab in the ECE 375 series and it covers the setup and compilation of an AVR Assembly Program. The student will learn how how to use the sample Basic Bump Bot assembly file and send the binaries to the AVR Microcontroller board. For the second part of the lab the student will be expected to download and compile the included C sample program and from it learn how to configure the I/O ports of the ATmega32U4 Microcontroller. The student will then write their own C program and upload it to the Microcontroller to verify that it runs as expected. The provided programs have been attached in the source code section of this report.

\section{Design}
As for part 1 of this lab assignment, no design needs to be done as the program is supplied. For part 2 of this lab assignment the C program was created to mimic the operaitons of the bump bot assembly file. Firstly the student must understand how the Bump Bot code must operate and they gain this information from the slides provided as they must program the right LED's to illuminate. For our program we decided that we wanted everything to be as readable as possible, thus we created constants for each of the LED directional cues. 
%TODO ADD BLOCK DIAGRAM FOR WHAT IS EXPECTED

\section{Assembly Overview}
As for the Assembly program an overview can be seen below


\subsection{Internal Register Definitions and Constants}
Four different registers have been setup, those being the multipurpose register (mpr), the wait counter register (waitcnt), and two loop counters, for counting the cycles of the delay function. In addition to these, there are several different constants. WTime defines the time  in milliseconds to wait inside the wait loop. The rest of the defined constants are either input bits, engine enable bits, or engine direction bits.

\subsection{Initialization Routine}
The initialization routine sets up several important ports and pointers that allow the rest of the assembly to work. Firstly the stack pointer is initialized at the end of RAM so that when the program pushes and pops items into and out of it, the stack does not interfere with any other data. Port B is then initialized for output, and Port D is initialized for input. The move forward command is also in this phase, to give a default movement type.

\subsection{Main Routine}
The main program constantly checks for if either of the whisker buttons have been hit, by reading the input of the PIND. When one of the whiskers is hit, the correct subroutine is called. As long as no button is hit the bump bot will continue in a straight line.

\subsection{Subroutines}
	\subsubsection{Hit Right}
	The HitRight subroutine describes what happens when the right whisker bit is triggered. The robot will move backwards for a second, then turn left for a second, then it will continue forward. 
	
	\subsubsection{Hit Left}
	The HitLeft subroutine describes what happens when the left whisker bit is triggered. First the bump bot will move backwards for a second, then it will turn right for a second, then it will continue forward. 
	
	\subsubsection{Wait}
	The Wait subroutine controls the wait intervals while the bump bot is preforming an action. Due to each clock cycle taking a measurable amount of time, we can calculate how many times we need to loop for. This function used the olcnt and ilcnt to have two nested loops, running the dec command until they equal zero, thus waiting the requested amount of time.

\section{C Program Overview}
Each of the methods determined to operate the bump bot can be seen in the code section at the end of this report, their descriptions are here.

\subsection{Definitions and Constants}
Several different constant integer values are prescribed on lines 29 - 33

\subsection{Main Method}
Text goes here

\subsection{Functions}
	\subsubsection{Hit Right}
	This is just an example.
	
	\subsubsection{Hit Left}
	Replace with your owns.



\section{Testing}
Text and Figures go here.
\begin{table}[h]
	\begin{tabular}{|l|l|l|ll}
		\cline{1-3}
		Case & Expected & Actual meet expected &  &  \\ \cline{1-3}
		&          &                      &  &  \\ \cline{1-3}
		&          &                      &  &  \\ \cline{1-3}
		&          &                      &  &  \\ \cline{1-3}
	\end{tabular}
\end{table}

\section{Additional Questions}
\begin{enumerate}
    \item
    The text of the question

    The text of the answer

    \item
    The text of the question
    \begin{enumerate}
        \item
        Text of the first part of the answer

        \item
        Text of the second part of the answer
    \end{enumerate}

\end{enumerate}

\section{Difficulties}
Text goes here

\section{Conclusion}
Text goes here

\section{Source Code}%


\lstinputlisting
[
caption=Assembely Bump Bot Script,
language={[x86masm]Assembler},
numbers =left,
rulesepcolor=\color{blue}
]{../Lab1/Lab1/main.asm}
\lstinputlisting[
caption=C Bump Bot Script,
language=C,
numbers =left,
rulesepcolor=\color{magenta}
]{../Lab1C/Lab1C/main.c}



\end{document}
